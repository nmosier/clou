\documentclass{article}

\usepackage{url}
\usepackage{graphicx}
% \usepackage{times}
\usepackage{tikz}
\usetikzlibrary{matrix,decorations.text,decorations.pathmorphing,calc,arrows,decorations.markings}
\usepackage{pgfplots}
\usepackage{amsmath}
\usepackage{graphicx}
\usepackage{float}
\usepackage{listings}
\usepackage[linesnumbered,lined]{algorithm2e}
\usepackage{algorithmic}
\usepackage{amsthm}
\usepackage{thm-restate}
\usepackage{tablefootnote}
\usepackage{multirow,multicol}
\usepackage[activate={true,noncompatibility}]{microtype}
\usepackage{units}
\newtheorem{theorem}{Theorem}[equation]
\newtheorem{lemma}[theorem]{Lemma}
\newtheorem{proposition}[theorem]{Proposition}
\newtheorem{definition}[theorem]{Definition}
\newtheorem{corollary}[theorem]{Corollary}
\newtheorem{conjecture}[theorem]{Conjecture}
\newtheorem*{conjecture*}{Conjecture}
\newtheorem*{problem}{Problem}
\newtheorem{claim}[theorem]{Claim}
\theoremstyle{definition}
\newtheorem*{remark}{Remark}
\newtheorem*{example}{Example}

%% If creating a pdf for digital viewing, uncomment the following:
\usepackage{xcolor}
\definecolor{darkred}  {rgb}{0.5,0,0}
\definecolor{darkblue} {rgb}{0,0,0.5}
\definecolor{darkgreen}{rgb}{0,0.5,0}
% Color links
\usepackage[colorlinks = true]{hyperref}
\hypersetup{
  urlcolor   = red,         % color of external links
  linkcolor  = blue,     % color of internal links
  citecolor  = darkgreen    % color of file links
}

% Clever references
\usepackage[nameinlink]{cleveref} % captialise
\crefname{lemma}{Lemma}{Lemmas}
\crefname{proposition}{Proposition}{Propositions}
\crefname{definition}{Definition}{Definitions}
\crefname{theorem}{Theorem}{Theorems}
\crefname{conjecture}{Conjecture}{Conjectures}
\crefname{corollary}{Corollary}{Corollaries}
\crefname{section}{Section}{Sections}
\crefname{appendix}{Appendix}{Appendices}
\crefname{figure}{Fig.}{Figs.}
\crefname{equation}{Eq.}{Eqs.}
\crefname{table}{Table}{Tables}
\crefname{claim}{Claim}{Claims}
\crefname{chapter}{Chapter}{Chapters}

\newcommand{\po}{\texttt{po}}
\newcommand{\com}{\texttt{com}}
\newcommand{\comx}{\texttt{comx}}
\newcommand{\tfo}{\texttt{tfo}}
\newcommand{\tfodepth}{\texttt{tfo\_depth}}
\newcommand{\preds}{\texttt{preds}}
\newcommand{\succs}{\texttt{succs}}
\newcommand{\nodes}{\texttt{nodes}}

\title {LCM Fence Synthesis Specification}

\author {Nicholas Mosier}

\begin{document}

\maketitle
\pagenumbering{roman}

\section{Relations}
\po \\
\tfo \\

Assume that
$\exists! \top \in \nodes \mid \preds [\top] = \emptyset$
and
$\exists! \bot \in \nodes \mid \succs [\bot] = \emptyset$.

\section{Constraints}
\subsection{\po}

A boolean variable $\po [N]$ is created for each AEG node $N$ representing whether this node is architecturally executed in the current execution.

\begin{itemize}
\item $po [\top]$
% \item $po [\bot]$
\item $\forall N \in \nodes \setminus \bot, \bigoplus_{M \in \succs [N]} \po [M]$
\item $\forall N \in \nodes \setminus \top, \bigvee_{M \in \preds [N]} \po [M]$
\end{itemize}

\subsection{\tfo}

% Track depth and whether it's tfo
An integer variable $\tfodepth [N]$ is created for each AEG node $N$ representing the depth from the most recent speculative fork.

\begin{itemize}
\item $\left( \tfo [N] \implies \neg \po [N] \right)$
\item $\left| \left\{ M \in \succs [N] \mid \tfo [M] \right\} \right| \leq 1$
\end{itemize}

\end{document}

%%% Local Variables:
%%% mode: latex
%%% TeX-master: t
%%% End:
